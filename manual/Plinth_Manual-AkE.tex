\documentclass{article}
\usepackage{graphicx}
\usepackage{hyperref}
\hypersetup{%
	pdfborder = {0 0 0}
}

\title{Plinth Manual}
\author{Tobias Brodel - Audiokinetic Experiments Lab, RMIT}

\begin{document}
	\pagenumbering{gobble}
	\maketitle
	\newpage
	\pagenumbering{gobble}
	\tableofcontents
	\newpage
	\pagenumbering{arabic}

	\section{Introduction}
	This piece of equipment, known as \emph{The Plinth,} is an audioreactive, 
	MIDI-programmable lighting design unit. Initially conceived by Darrin 
	Verhagen and built by Stuart McFarlane it consists of a strip of 34 
	RGBLEDs, controlled by an Arduino and driven by Pure Data. It is 
	designed to produce tight, syncretic audiovisual experiences with a mininum 
	of latency. The composer is not expected to be a programmer, all it takes 
	to operate this instrument is a basic knowledge of MIDI. The reference 
	implementation in AkE uses Ableton Live but in theory any MIDI-compliant 
	DAW or instrument should be able to interface with \emph{The Plinth.} Bug 
	reports are welcome.

	\section{Getting Started}
	\subsection{Setup}
	\emph{The Plinth} is intended to be a modular system, able to plug and play 
	with a wide variety of hardware and software. However before we can get our 
	hands on those sweet blinkenlights we must perform a few setup tasks. You 
	will want to download ``plinth-setup.zip" from [link].

	\subsubsection{Saffire Setup}
	First you'll need to setup the Focusrite Saffire PRO 40 soundcard attached 
	the workstation in AkE. Open the .zip you downloaded and locate a file 
	called ``plinth-saffire.pro40". By opening this file you should have loaded 
	the appropriate settings for the soundcard. The main things to check are 
	that DAW1 and DAW2 are routed to Line Outputs 7 and 8 respectively and that 
	DAW9 and DAW10 are going to Line Outputs 9 and 10. This will ensure that 
	the mix gets to the headphones (7/8) and isolated audio sends can reach the 
	plinth (9/10).

	\subsubsection{Using Live}
	Inside the ``plinth-setup" folder there should be a ``plinth-live Project" 
	Ableton Live session. Open it on the workstation in AkE, then open 
	preferences (Cmd+,). Navigate to the Audio tab and select ``Saffire (20 in, 
	20 out)" for both ``Audio Input Device" and ``Audio Output Device". 
	Remaining in the ``Audio" tab, clink the ``Output Config" button and enable 	all sends from 1-10. Now navigate to the MIDI tab and enable ``Output: 
	Pro40(MIDI)" by clicking the ``Off" button under the ``Track" column. It 
	should now be highlighted yellow and read ``On".
	
	In the session you should find six MIDI tracks:
	\begin{enumerate}
		\item Pixel
		\item Routine
		\item Luminosity
		\item Delay
		\item Pixel Method
		\item Luminosity Method
	\end{enumerate}
	Let's not worry too much about what they do for now. First, make sure the 
	workstation MIDI is connected to \emph{The Plinth.} Now set \emph{Routine} 
	to 1, \emph{Luminosity} to 127 and play some notes around middle-C (60) on 
	the \emph{Pixel} track. If you see lights then you're in business.

	There's also an audio track, make sure that is being sent to channels 9/10 
	on the Saffire.

	\subsubsection{Using Pro Tools}
	Another option is to open the ``plinth-protools" session. Check under 
	``Setup-$>$Playback Engine" that ``Saffire" is selected. Then open 
	``Setup-$>$I/O" go through every tab clicking the ``Default" button. The 
	tracks should be set up identically to Ableton Live.

	\subsubsection{Using Other DAWs}
	If you're using something other than Ableton Live or Pro Tools then you 
	have to set these tracks up yourself. You'll need one regular MIDI signal 
	on channel 1 to voice notes from and five control-change (CC-data) signals 
	to control the other parameters of the instrument. These CC signals need to 
	communicate on channels 1-5 respectively and use parameter 6 (Data Entry). 
	
	In addition you will need a stereo audio send. In the examples above I've 
	isolated that from the main mix. I'm sending it out from channels 9/10 on 
	the AkE workstation, if you're using your own hardware or know what you're 
	doing then any send \emph{should} work.

	\subsubsection{MIDI-only Setups}
	Some setups may restrict you to a MIDI-only situation. In this case simply 
	make sure that \emph{Pixel Method} is set to 0 and \emph{Luminosity Method} 
	is set to between 0 and 83 for reasons soon to be elucidated.

	\subsection{Controls}
	Now you have the MIDI set up, let's test that it does what it says on the 
	tin.

	\subsubsection{Pixel}
	The \emph{Pixel} track should select a pixel to illuminate. It is triggered 
	by sending a note between 60 and 93. Depending on what the 
	\emph{Pixel Method} track is set to it may do nothing, make sure that this 
	track is set to 0.

	\subsubsection{Routine}
	This track switches between one of seventeen behaviours programmed into the 
	Arduino. Making sure that CC data is being sent over channel 1 with the 
	parameter set to 6 the LEDs should respond to this signal. A value of 0 
	should turn off all LEDs, a value of 1 should illuminate one pixel and a 
	value of 10 should turn all lights on to the same colour.

	\subsubsection{Luminosity}
	CC data sent on channel 2, parameter 6 should control the luminosity, the 
	brightness of the LEDs. This is dependent on the setting of the 
	\emph{Luminosity Method} channel, make sure that is set to 0.

	\subsubsection{Delay}
	Some routines respond to a \emph{Delay} argument, most however do not. At 
	the moment this breaks most things, I would recommend leaving it set to 0. 
	This CC signal runs over channel 3, parameter 6.

	\subsubsection{Pixel Method}
	This track decides how pixels are selected, with the \emph{Pixel} MIDI 
	track, or via audio. If set between 0-63 the \emph{Pixel} track will select 
	LEDs like usual, otherwise (64-127) audio amplitude data will determine 
	the pixel. Communicates on channel 4, parameter 6.

	\subsubsection{Luminosity Method}
	Much like the above channel, this signal controls the way luminosity is 
	manipulated. 0-42 uses the \emph{Luminosity} track, 43-83 uses velocity 
	data from the \emph{Pixel} track and 84-127 uses audio amplitude data. 
	Communicates on channel 5, parameter 6.

	\subsection{The Pixel Parameter}
	Astute readers will have surmised that there are four parameters being 
	controlled here. \emph{Routine, Luminosity} and \emph{Delay} are reasonably 
	obvious, however \emph{Pixel} bears some explanation.

	The strip of LEDs on the plinth is an array of 34 pixels, each capable of 
	producing any 24-bit RGB colour. By default each LED is assigned a default 
	base hue which informs the behaviours of any given routine. There are 
	twelve base colours which traverse the visible electromagnetic spectrum in 
	a clockwise fashion, from 0-33  in groups of three:
	\begin{itemize}
		\item Pixels 0-2: Red
		\item Pixels 3-5: Orange
		\item Pixels 6-8: Yellow
		\item Pixels 9-11: Chartreuse
		\item Pixels 12-14: Green
		\item Pixels 15-17: Green-Cyan
		\item Pixels 18-20:	Cyan
		\item Pixels 21-23: Blue-Cyan
		\item Pixels 24-26:	Blue
		\item Pixels 27-29:	Violet
		\item Pixels 30-32: Magenta
		\item Pixel 33: Red-Magenta
	\end{itemize}

	\section{The Routines}
	This section of the manual details the bahaviour of the seventeen routines 
	programmed into the Arduino.

	\subsection{Routine 0: Off}
	This routine turns all LEDs off, regardless of any other parameters.

	\subsection{Routine 1: Single}
	This routine illuminates a single pixel. Its luminosity is variable. 
	Unaffected by \emph{Delay.}

	\subsection{Routine 2: Spectrum}
	Turns all LEDs on at their base colour. Luminosity variable. Unaffected by 
	\emph{Delay.}

	\subsection{Routine 3: Spectrum Chase}
	Traverses entire plinth in clockwise sequence until all LEDs are 
	illuminated, starting with \emph{Pixel}. Luminosity variable. The interval 
	between pixels can be varied using \emph{Delay} (experimental).

	\subsection{Routine 4: Double Spectrum Chase}
	Similar to above but travels in both directions around the circle, starting 
	with \emph{Pixel.} Luminosity Variable. Affected by \emph{Delay} 
	(experimental).

	\subsection{Routine 5: Diad}
	Illuminates LED at \emph{Pixel} and LED directly opposite in their 
	respective base colours. Luminosity variable. Unaffected by \emph{Delay.}

	\subsection{Routine 6: Triad}
	Illuminates equilateral triangle based upon LED at \emph{Pixel} at each 
	each pixel's default colour. Luminosity variable. Unaffected by 
	\emph{Delay.}

	\subsection{Routine 7: Triad Chase}
	Illuminates a series of triangles of increasing distance, based upon LED at 
	\emph{Pixel} at each pixel's default colour. Luminosity variable. Affected 
	by \emph{Delay} (experimental).

	\subsection{Routine 8: Tetrad}
	Illuminates four LEDs in equal distances, based on LED at \emph{Pixel} at 
	each pixel's default colour. Luminosity variable. Unaffected by 
	\emph{Delay.}

	\subsection{Routine 9: Skinny Tetrad}
	Illuminates four LEDs in unequal distances, based on LED at \emph{Pixel} at 
	each pixel's default colour. Luminosity variable. Unaffected by 
	\emph{Delay.}

	\subsection{Routine 10: Mono}
	Illuminates all pixels to the colour of \emph{Pixel.} Luminosity variable. 
	Unaffected by \emph{Delay.}

	\subsection{Routine 11: Hemisphere}
	Illuminates all pixels. Half beginning at \emph{Pixel} set to that LED's 
	base colour. Other half set to its complementary. See \emph{Diad.} 
	Luminosity variable. Unaffected by \emph{Delay.}

	\subsection{Routine 12: Trisect}
	Like \emph{Hemisphere} but split into thirds. See \emph{Triad}. Luminosity 
	variable. Unaffected by \emph{Delay.}

	\subsection{Routine 13: Quadrant}
	Like previous two routines but split into quarters. See \emph{Tetrad.} 
	Luminosity variable. Unaffected by \emph{Delay.}


	\subsection{Routine 14: Outsider}
	All LEDs illuminated at one eighth of given luminosity. Colour set by 
	default at \emph{Pixel}. Another LED chosen at random is set to 
	complementary colour (see \emph{Diad}) and illuminated at given luminosity. 
	Affected by \emph{Delay} (experiemntal).

	\subsection{Routine 15: Outsider Triad}
	Like \emph{Outsider} but with two random pixels. Hues set as in 
	\emph{Triad.} Luminosity variable. Affected by \emph{Delay} (experimental).

	\subsection{Routine 16: Outsider Tetrad}
	Like previous two routines but with three random pixels. See \emph{Tetrad.} 
	Luminosity variable. Affected by \emph{Delay} (experimental).

	\section{Coding}
	The sources to all of the software running the plinth are available at 
	\url{https://github.com/TobyJamesJoy/ake-plinth}. There are three main 
	elements to the code, the Arduino sketch, the PD patches and this manual. 
	Currently work is underway to build a virtual model of the plinth in the 
	Unity game engine so students can develop works offsite. Everything is BSD 
	licensed, feel free to check out the code, patches are welcome.

	\subsection{Arduino}
	The Arduino code is written in Kernel Normal Form (KNF, see 
	\url{http://www.openbsd.org/cgi-bin/man.cgi/OpenBSD-current/man9/style.9}).
	This is an attempt to keep the code conformant to an ANSI-C style. I want 
	to avoid C++isms so as to make everything portable. In the future we may 
	choose to move away from Arduino to for instance plain C running on an 
	Atmel chip or something like a Raspberry Pi. Also KNF is the only sane 
	coding style ever. The code links to ``LPD8806.h", found at 
	\url{https://github.com/adafruit/LPD8806} and ``SPI.H" that ships with the 
	Arduino IDE.

	\subsection{Pure Data}
	The PD component of this comprises of two patches, ``daw\_ctl.pd" and 
	``arduino.pd", they do pretty much what it says on the tin. ``daw\_ctl.pd" 
	accepts the MIDI, CC and audio signals from the workstation's DAW and 
	parses them. The results are then handed to ``arduino.pd" in the form of 
	four sends: ``rout", ``pix", ``lum" and ``del". ``arduino.pd" parses this 
	data again and communicates it via serial bus to Arduino to generate the 
	routines.
	
	If you're hacking on this at home please be mindful that the plinth runs 
	vanilla PD from the Debian repos, this again is for portability's sake. In 
	the future we'll probably having this running from a low-cost SoC, for 
	which there probably won't be pd-extended binaries. As a result many 
	objects rely on the maxlib, cyclone and comport externals. If you're 
	testing on Debian install everything like so:
	\begin{verbatim}
	apt-get install puredata pd-comport pd-cyclone pd-maxlib
	\end{verbatim}
	I find it's good practice to know which objects are from vanilla and which 
	are external so I don't start PD with any -lib flags. As such you'll need 
	to specify the library name when declaring new objects (e.g. 
	``cyclone/drunk" and ``maxlib/scale").

	\subsection{Manual}
	The source for this manual too is available on github, it's written in 
	\LaTeX using the texlive package from the Debian repos. If any of this 
	could be clearer or if you think I've left something out, please send me a 
	patch!
\end{document}
